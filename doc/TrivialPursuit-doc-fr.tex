% !TeX TXS-program:compile = txs:///pdflatex

\documentclass[11pt,a4paper]{ltxdoc}
\usepackage{TrivialPursuit}
\usepackage{fancyvrb}
\usepackage{fancyhdr}
\fancyhf{}
\renewcommand{\headrulewidth}{0pt}
\lfoot{\sffamily\small [TrivialPursuit]}
\cfoot{\sffamily\small - \thepage{} -}
\rfoot{\hyperlink{matoc}{\small\faArrowAltCircleUp[regular]}}
\usepackage{hologo}
\providecommand\tikzlogo{Ti\textit{k}Z}
\providecommand\TeXLive{\TeX{}Live\xspace}
\providecommand\PSTricks{\textsf{PSTricks}\xspace}
\let\pstricks\PSTricks
\let\TikZ\tikzlogo

\usepackage{hyperref}
\urlstyle{same}
\hypersetup{pdfborder=0 0 0}
\usepackage[margin=2cm]{geometry}
\setlength{\parindent}{0pt}
\def\TPversion{0.1.0}
\def\TPdate{24 novembre 2023}
\usepackage{tcolorbox}
\tcbuselibrary{skins,hooks,listingsutf8}
\usepackage{soul}
\sethlcolor{lightgray!25}
\NewDocumentCommand\MontreCode{ m }{%
	\hl{\vphantom{\texttt{pf}}\texttt{#1}}%
}

\begin{document}

\pagestyle{fancy}

\thispagestyle{empty}

\begin{center}
	\begin{minipage}{0.88\linewidth}
	\begin{tcolorbox}[colframe=yellow,colback=yellow!15]
		\begin{center}
			\begin{tabular}{c}
				{\Huge \texttt{TrivialPursuit [fr]}}\\
				\\
				{\LARGE Créer un plateau type '\textit{Trivial Pursuit}',} \\
				{\LARGE avec personnalisations.} \\
				\\
				{\small \texttt{Version \TPversion{} -- \TPdate}}
		\end{tabular}
		\end{center}
	\end{tcolorbox}
\end{minipage}
\end{center}

\begin{center}
	\begin{tabular}{c}
	\texttt{Cédric Pierquet}\\
	{\ttfamily c pierquet -- at -- outlook . fr}\\
	\texttt{\url{https://github.com/cpierquet/trivialpursuit}} \\
	\\
	\texttt{Trivial Pursuit\texttrademark, est une marque déposée par Hasbro.}
\end{tabular}
\end{center}

\hrule

\vfill

\begin{tcolorbox}[colframe=lightgray,colback=lightgray!5]
\PlateauTrivialPursuit
\end{tcolorbox}

\vfill~

\pagebreak

\phantomsection

\hypertarget{matoc}{}

\tableofcontents

\vspace*{5mm}

\hrule

\vspace*{5mm}

\section{Introduction}

\subsection{Description}

L'idée de ce package est de proposer un plateau de jeu de type \textsf{Trivial Pursuit} (marque déposée par Hasbro) avec personnalisations au niveau :

\begin{itemize}
	\item des dimensions (rayon + hauteur bordure) ;
	\item des couleurs ;
	\item des logos (sous forme de caractères, comme \MontreCode{fontawesome5}).
\end{itemize}

L'alternance des cases et des couleurs est reprise du plateau \textit{classique}, donc est fixée globalement.

\medskip

Pour le moment, les \textit{limitations} sont :

\begin{itemize}
	\item le nombre de catégories est (pour le moment) fixé à 6 ;
	\item les logos sont des caractères, comme ceux du package \MontreCode{fontawesome5}.
\end{itemize}

\subsection{Chargement}

Pour charger le package, il suffit d'ajouter, dans le préambule :

\begin{quote}
\begin{verbatim}
\usepackage{TrivialPursuit}
\end{verbatim}
\end{quote}

Les packages chargés et utilisés par \MontreCode{TrivialPursuit} sont :

\begin{itemize}
	\item \MontreCode{tikz} avec les librairies \MontreCode{calc,positioning} ;
	\item \MontreCode{calc} et \item \MontreCode{fontawesome5} ;
	\item \MontreCode{simplekv} ; \MontreCode{xintexpr} et \MontreCode{listofitems}.
\end{itemize}

\vfill

\subsection{Historique}

\begin{quote}
\begin{verbatim}
0.1.0 : Version initiale
\end{verbatim}
\end{quote}

\pagebreak

\section{La commande}

\subsection{Fonctionnement global}

Le code se charge de tracer le plateau avec les éventuels \textit{ornements}.

Toutes les clés disponibles sont présentées dans le paragraphe suivant, mais concernant les dimensions, des \textit{générales} sont proposées par défaut, mais il est toutefois possible de les modifier :

\begin{itemize}
	\item soit explicitement par les clés \MontreCode{[Rayon=]} et \MontreCode{[HauteurCases=]} ;
	\item soit en spécifiant une unité particulière grâce à la clé \MontreCode{[Unite=]}, au quel cas la forme générale du plateau sera conservée.
\end{itemize}

\begin{tcblisting}{colframe=lightgray,colback=lightgray!5}
\PlateauTrivialPursuit[Unite=0.5]
%
\PlateauTrivialPursuit[Rayon=4,HauteurCases=1.25] %attention au rendu...

\PlateauTrivialPursuit[Unite=0.33] %le rendu est 'optimal'
\end{tcblisting}

\pagebreak

\subsection{Clés et options}

Les \textsf{clés} disponibles sont :

\begin{itemize}
	\item \MontreCode{Rayon} := rayon du plateau (en unité), et valant \MontreCode{8} par défaut ;
	\item \MontreCode{HauteurCases} := hauteur des cases extérieures, et valant \MontreCode{1.5} par défaut ;
	\item \MontreCode{ListeCouleurs} := couleurs des catégories (sens trigo),
	
	\hfill{}et valant \MontreCode{yellow,orange,blue,brown,green,pink} par défaut ;
	\item \MontreCode{ListeIcones} := icônes des catégories (sens trigo),
	
	\hfill\MontreCode{\textbackslash faArchway,\textbackslash faTableTennis,\textbackslash faUmbrellaBeach,\textbackslash faGuitar,\textbackslash faTree,\textbackslash faMusic} par défaut ;
	\item \MontreCode{Icones} := booléen pour afficher les icônes, et valant \MontreCode{true} par défaut ;
	\item \MontreCode{Centre} := booléen pour colorer l'hexagone centrale, et valant \MontreCode{true} par défaut ;
	\item \MontreCode{CouleurCentre} := couleur du centre hexagonal, et valant \MontreCode{lightgray!25} par défaut ;
	\item \MontreCode{Couleurs} := booléen pour afficher les couleurs, et valant \MontreCode{true} par défaut ;
	\item \MontreCode{Logo} := booléen pour afficher un logo central, et valant \MontreCode{true} par défaut ;
	\item \MontreCode{Jokers} := booléen pour afficher les jokers, et valant \MontreCode{true} par défaut ;
	\item \MontreCode{IconeJoker} := icône des jokers, et valant \MontreCode{\textbackslash faDice} par défaut ;
	\item \MontreCode{Vide} := booléen pour afficher le plateau vide, et valant \MontreCode{false} par défaut ;
	\item \MontreCode{Unite} := pour préciser l'unité (et gérer un facteur d'échelle), et valant \MontreCode{1} par défaut ;
	\item \MontreCode{Epaisseur} := épaisseur des tracés, et valant \MontreCode{0.8pt} par défaut ;
	\item \MontreCode{Rotation} := rotation éventuelle du plateau, et valant \MontreCode{0} par défaut.
\end{itemize}

\begin{tcblisting}{colframe=lightgray,colback=lightgray!5}
\PlateauTrivialPursuit[Rotation=35,Unite=0.5]
\PlateauTrivialPursuit[Unite=0.5]
\end{tcblisting}

\begin{tcblisting}{colframe=lightgray,colback=lightgray!5}
\PlateauTrivialPursuit[Unite=0.5,Icones=false]
\PlateauTrivialPursuit[Unite=0.5,Couleurs=false]
\end{tcblisting}

\begin{tcblisting}{colframe=lightgray,colback=lightgray!5}
\PlateauTrivialPursuit[Unite=0.5,Vide]
\PlateauTrivialPursuit[Unite=0.5,Logo=false,Centre=false,IconeJoker=\faAward]
\end{tcblisting}

\begin{tcblisting}{colframe=lightgray,colback=lightgray!5}
\PlateauTrivialPursuit[%
	Jokers=false,%
	ListeCouleurs={%
		blue!25,red!25,teal!25,orange!25,gray!50,violet!25},%
	ListeIcones={
		\faAddressCard,\faAngleDoubleRight,\faAngry[regular],%
		\faAtom,\faBalanceScaleLeft,\faBell}
]
\end{tcblisting}
\end{document}